\documentclass[a4paper, fontsize=11pt]{scrartcl} % A4 paper and 11pt font size
% \usepackage{showframe} % page debugging
\usepackage[bottom=1.0in]{geometry} % slightly shorter margin bottom
\usepackage{times}
\usepackage[T1]{fontenc} % use 8 bit encoding that has 256 glyphs
\usepackage[english]{babel} % English language/hyphenation
\usepackage{csquotes}
\usepackage{parskip}
\usepackage{sectsty}
\usepackage{enumitem}
\usepackage{url}
\usepackage[
  backend=biber,
  style=alphabetic,
  citestyle=authoryear
]{biblatex}
\addbibresource{aero_report.bib}

\allsectionsfont{\normalfont\scshape}
\setlength{\headheight}{7.6pt}
\setlength{\headsep}{0pt}
\setlength{\parsep}{0pt}
\setlength{\topskip}{0pt}
\setlength{\topmargin}{0pt}
\setlength{\topsep}{0pt}
\setlength{\partopsep}{0pt}

%--------------------------------------------------------------------------------
%   TITLE SECTION
%--------------------------------------------------------------------------------

\newcommand{\horrule}[1]{\rule{\linewidth}{#1}}

\title{
        \normalfont \normalsize
        \textsc{ucl, computer science}
        \horrule{0.5pt} \\[0.4cm]
        \huge{Case Study: Junkers 87 'Stuka'}\\
        \horrule{0.5pt} \\[0.5cm]
}

\author{Ben Ryves}

\date{\normalsize\today}

\begin{document}
\maketitle

% abstract



% development of bombers - need for precision
% development of dive bombers

The first attack launched from an airplane was undertaken by Giulio
Gavotti in 1911, during the Italo-Turkish war, when Gavotti dropped
grenades from his Taube Monoplane \autocite{bbc11}. A few years later,
at the start of World War 1 (WW1), Italy and Russia had developed purpose
built bombers, and by the end of WW1 most of the belligerent nations
had developed bombing capabilities of their own (citation), using
aerodynamically shaped bombs with no guidance or thrust. When dropped from a
high altitude during horizontal strafes of the target, these bombs had
complicated trajectories which were affected by drag and gravity, and
which were sighted using fixed sights to provide an estimated impact point
without accounting for the prevailing atmospheric conditions
(citation). Due to these conditions, bombs had a large circular error
probable (CEP) \autocite{cep88}, limiting bombing to mass
interdiction strategies requiring air supremacy.

The dive-bomber design emerged as a solution to the need for precision
bombing of strategic targets, and enabled combined arms close air
support for ground units. Diving towards the target simplified
calculation of the trajectory of a dropped bomb since the distance
from release to impact was reduced, and the inital velocity was now
acting in the direction of the target. The first preplanned dive bombing
attack was carried out in 1918 against a German supply barge in France,
sinking the barge with a 20-pound Cooper bomb \autocite[p.~72]{boyne10},
and in the interwar period a number of designs were produced. (cont here)

The Junkers Ju 87 'Stuka', was a dive bomber developed by Hermann
Pohlmann in 1933, and which was first flown in 1935 \autocite[p.~9]{weal97}.

Unusually for a strategic aircraft in World War II, the Stuka had fixed
landing gear.\autocite[p.~4]{curry88}

When calculating operational parameters for the Stuka, it is necessary
to consider that the Luftwaffe deployed the Stuka in the Western and
Eastern fronts, as well as the Desert and Meditteranean theatres.
Accordingly, calculations have been done for the most extreme
temperatures where the Stuka operated, as well as the more temperate
Battle of Britain temperatures, in order to demonstrate the versatility
of the Stuka.

Lowest temperature of operation (Moscow): -45C \autocite{raus03}
European / Battle of Britain temperatures:
Highest temperature of operation ():

% calculate atmosphere for Eastern, Western, Desert theatres, over land
% as well as sea

\printbibliography
\end{document}

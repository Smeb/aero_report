\documentclass[a4paper, fontsize=11pt]{scrartcl} % A4 paper and 11pt font size
% \usepackage{showframe} % page debugging
\usepackage[bottom=1.0in]{geometry} % slightly shorter margin bottom
\usepackage{times}
\usepackage[T1]{fontenc} % use 8 bit encoding that has 256 glyphs
\usepackage[english]{babel} % English language/hyphenation
\usepackage{csquotes}
\usepackage{parskip}
\usepackage{sectsty}
\usepackage{enumitem}
\usepackage{url}
\usepackage[
  backend=biber,
  style=alphabetic,
  citestyle=authoryear
]{biblatex}
\addbibresource{aero_report.bib}

\allsectionsfont{\normalfont\scshape}
\setlength{\headheight}{7.6pt}
\setlength{\headsep}{0pt}
\setlength{\parsep}{0pt}
\setlength{\topskip}{0pt}
\setlength{\topmargin}{0pt}
\setlength{\topsep}{0pt}
\setlength{\partopsep}{0pt}

%--------------------------------------------------------------------------------
%   TITLE SECTION
%--------------------------------------------------------------------------------

\newcommand{\horrule}[1]{\rule{\linewidth}{#1}}

\title{
        \normalfont \normalsize
        \textsc{ucl, computer science}
        \horrule{0.5pt} \\[0.4cm]
        \huge{Case Study: Junkers 87 'Stuka'}\\
        \horrule{0.5pt} \\[0.5cm]
}

\author{Ben Ryves}

\date{\normalsize\today}

\begin{document}
\maketitle

% abstract



% development of bombers - need for precision
% development of dive bombers
\section{The Case for Dive Bombing}

The first attack launched from an airplane was undertaken by Giulio
Gavotti in 1911, during the Italo-Turkish war, when Gavotti dropped
grenades from his Taube Monoplane \autocite{bbc11}. A few years later,
at the start of World War 1 (WW1), Italy and Russia had developed purpose
built bombers, and by the end of WW1 most of the belligerent nations
had developed bombing capabilities of their own (citation), using
aerodynamically shaped bombs with no thrust or guidance systems ('dumb'
bombs). When dropped from a
high altitude during horizontal strafes of the target, these bombs had
complex trajectories which were affected by drag and gravity, and
which were sighted using fixed sights to provide an estimated impact point
without accounting for the prevailing atmospheric conditions
(citation). Due to these conditions, munition accuracy, expressed in
terms of the  circular error probable (CEP) \autocite{cep88}, the
circular region in which 50\% of munitions were predicted to land, was
poor [more needed], and bombing was limited to mass interdiction strategies
requiring air supremacy in order to be successful.

The dive-bomber design emerged as a solution to the need for precision
bombing of tactical objectives during the later stages of
WW1, and was eventually replaced by improvements in bomb sighting,
increased bomb payloads, and guided weaponry. Dive bombers would align themselves
laterally before engaging in a dive towards a target,
descending to a given height before releasing their payload and pulling
out of the dive. Diving aligned the velocity of the aircraft in the
direction of the target, and reduced the distance travelled from release
to impact; this significantly reduced targeting complexity, and
correspondingly shrank the CEP of the payload. Additionally, the steep angle of dives
meant that the target remained in view of the pilot at the
point of payload release, further increasing the accuracy that could be
achieved.

The increased accuracy offered by dive bombing was tactically
significant, enabling both close air support
of ground forces in combined arms operations without risk to engaged
units, as well as accurate attacks against shipping which had been
difficult to accomplish with interdiction bombing; however, diving
towards targets placed aircraft at increased risk from surface fire, and
steep diving maneuvers limited payload weight and placed increased
stress on the craft. Dive bombers were also targets of opportunity for
enemy fighters, since their diving manoeuvres were predictable and broke from
protective formations, and they could not match the manoeuvrability or
speed of fighter craft; this, in addition to improvements in sighting
technology, lead to the decline of the dive bomber after WW2.

% weird ordering

In the interwar period (WW1 - WW2), purpose built dive bombers were constructed in
response to the success of initial dive bombing operations. However, due
to post-war treaties, German design of a dive bomber was
stymied by restrictions on rearmament and military development, and
German production of a dive bomber would not commence until 1932, when
the German Reichswehr (defense ministry) ordered development of a dive
bomber. Of the four large German aircraft companies, the contract for
the military dive bomber was actively pursued by two; Heinkel and
Junkers.

\section{Junkers Ju 87}

The Junkers Ju 87 'Stuka' design, the development of
which was lead by Hermann Pohlman, began development in 1933. First flown
in 1935 \autocite[p.~9]{weal97}, the Stuka succeeded a previous
Junkers Ju 47 K dive bombing design, which had been rejected by the
Reichswehr as too expensive, and replaced the
Heinkel He 50 as the dive bomber of choice for the Luftwaffe. The
Junkers was first flown in combat in 1938, in the Spanish civil war, and
was used throughout WW2 to great effect, first in the combined arms
'Blitzkrieg' tactics employed in Czechoslovakia, Poland, and France, as
well as in the bombing of shipping in the North, Mediterranean, and
Black Seas, and in support of Reich forces in Africa. In the latter
stages of the war, as the availablility of experienced pilots and the production capability of the Reich
dropped, and the design's weaknesses against fighters showed, the Stuka
was deployed only on the Eastern front, where the Luftwaffe still
maintained air supremacy, and was also used in nighttime operations in order to
reduce the potential risk to valuable pilots and aircraft.

% include image of Stuka

% materials
% shape
% function
% range, operational parameters
\subsection{Payload}
\section{General Design of the Ju 87}
\subsection{Structure \& Materials}
In 1915, the founder of Junkers, Hugo Junkers, had pioneered airplane
construction using a single anchored beam and all metal design, the
purpose of which was to improve the structural integrity of the aircraft
to the extent that no external support structures or bracing would be
required
\section{Testing, Production, Variants}
All Stuka variants featured reversed gull wings, a 'broken nose' [reference]
hump to the fore of the cockpit, and fixed landing gear with aerodynamic
spatting\autocite[p.~4]{curry88}, lending the
appearance of [quote, reference]. % reasons for these three design
% choices

As a wartime plane, the Stuka had a number of evolutionary variants as
Luftwaffe requirements changed during the course of the war. Principal among these were the Ju 87A
(pre-war prototype), Ju 87B (early war, Battle of Britain), Ju 87D
(late war, improved performace), and Ju 87G (anti tank close support
role).
\subsection{Ju 87A}
The 87A went through a number of iterations in the lead up to WW2,
until it was eventually replaced by the Ju 87B production design. 

{image of Ju 87A}
\subsection{Ju 87B}
\subsection{Ju 87D}
\subsection{Ju 87G}
\section{Aeronautic Performance}

\section{Role Performance}
When calculating operational parameters for the Stuka, it is necessary
to consider that the Luftwaffe deployed the Stuka in the Western and
Eastern fronts, as well as the Desert and Meditteranean theatres.
Accordingly, calculations have been done for the most extreme
temperatures where the Stuka operated, as well as the more temperate
Battle of Britain temperatures, in order to demonstrate the versatility
of the Stuka.

Lowest temperature of operation (Moscow): -45C \autocite{raus03}
European / Battle of Britain temperatures:
Highest temperature of operation ():

% calculate atmosphere for Eastern, Western, Desert theatres, over land
% as well as sea

\printbibliography
\end{document}

\documentclass[a4paper, fontsize=11pt]{scrartcl} % A4 paper and 11pt font size
% \usepackage{showframe} % page debugging
\usepackage[bottom=1.0in]{geometry} % slightly shorter margin bottom
\usepackage{times}
\usepackage[T1]{fontenc} % use 8 bit encoding that has 256 glyphs
\usepackage[english]{babel} % English language/hyphenation
\usepackage{csquotes}
\usepackage{parskip}
\usepackage{sectsty}
\usepackage{enumitem}
\usepackage{textcomp}
\usepackage{url}
\usepackage[
  backend=biber,
  style=alphabetic,
  citestyle=authoryear
]{biblatex}
\addbibresource{aero_report.bib}

\allsectionsfont{\normalfont\scshape}
\setlength{\headheight}{7.6pt}
\setlength{\headsep}{0pt}
\setlength{\parsep}{0pt}
\setlength{\topskip}{0pt}
\setlength{\topmargin}{0pt}
\setlength{\topsep}{0pt}
\setlength{\partopsep}{0pt}

%--------------------------------------------------------------------------------
%   TITLE SECTION
%--------------------------------------------------------------------------------

\newcommand{\horrule}[1]{\rule{\linewidth}{#1}}

\title{
        \normalfont \normalsize
        \textsc{ucl, computer science}
        \horrule{0.5pt} \\[0.4cm]
        \huge{Case Study: Junkers 87 'Stuka'}\\
        \horrule{0.5pt} \\[0.5cm]
}

\author{Ben Ryves}

\date{\normalsize\today}

\begin{document}
\maketitle

% abstract



% development of bombers - need for precision
% development of dive bombers
\section{Junkers Ju 87}

The Junkers Ju 87 'Stuka' design, the development of
which was lead by Hermann Pohlman, began development in 1933. First flown
in 1935 \autocite[p.~9]{weal97}, the Stuka succeeded a previous
Junkers Ju 47 K dive bombing design, which had been rejected by the
German Reichswehr (defense ministry) as too expensive, and replaced the
Heinkel He 50 as the dive bomber of choice for the Luftwaffe. Heinkel
had also been developing a competing design, the 118, but lost the dive
bomber contract when their design could not demonstrate the prequisite
ability to dive at a 90\textdegree\ angle, disintegrating during flight
\autocite[p.~68-69]{killen67} and forcing the pilot and judge of the
competition, Ernst Udet, to bail out.

The dive-bomber design emerged as a solution to the need for precision
bombing of tactical objectives during the later stages of
WW1, and was eventually replaced by improvements in bomb sighting,
increased bomb payloads, and guided weaponry. Dive bombers would align themselves
laterally before engaging in a dive towards a target,
descending to a given height to release their payload and then pull
out of the dive. Diving aligned the velocity of the aircraft in the
direction of the target, and reduced the distance travelled from release
to impact; this significantly reduced targeting complexity, and
correspondingly shrank the circular error probable (CEP), the
region in which 50\% of munitions were predicted to land. In comparison,
horizontal bombers using unguided bombs had to compensate for a parabolic
bomb trajectory where a bomb had horizontal velocity at launch and was
acted upon by drag and gravity during flight. Even in
calm weather, using tachometric bombsights, horizontal bombers during
WW2 had large CEPs and could not accurately hit small targets, being
more suited to mass scale interdiction bombing 'area-denial' sorties.

The increased accuracy offered by dive bombing was tactically
significant, enabling both close air support
of ground forces in combined arms operations without risk to engaged
units, as well as accurate attacks against shipping which had been
difficult to accomplish with interdiction bombing; however, diving
towards targets placed aircraft at increased risk from surface fire, and
steep diving maneuvers limited payload weight and placed increased
stress on the craft. Dive bombers were also targets of opportunity for
enemy fighters, since their diving manoeuvres were predictable and broke from
protective formations, and they could not match the manoeuvrability or
speed of fighter craft; this, in addition to improvements in sighting
technology, lead to the decline of the dive bomber after WW2.
\subsection{Payload}
\section{General Design of the Ju 87}
\subsection{Structure}
\subsubsection{Body}
The Stuka was a monoplane, with the fuselage acting as an
anchor for a single spar which passed through both wings and served to
distribute the load caused by lift.
This design, pioneered by Hugo Junkers in 1915, provided sufficient
strength to support a single fixed wing, and superseded
biplane designs by allowing for elimination of drag caused by interference
between the two wings, as well as drag caused by adding supporting structures
to the wings (such as struts). The Stuka's fuselage was constructed from two oval
sections, which were joined along the longitudinal axis to form the
chassis, and which was attached to series of U-shaped longerons running
the length of the longitudinal axis. These longerons were also attached
by curved brackets to frames which ran the length of the Stuka, and which were
arranged in a z-shaped pattern in order to facilitate inspection of
the longerons.
\subsubsection{Wings}

\subsection{Materials}

The body itself was made principally from
duralumin, an aluminum alloy composed from copper, magnesium, and
manganese\footnote{Unfortunately, precise values for the alloy could not
be located, but is likely to have been a composition of 95\% aluminum, 4\%
copper, 0.5 - 1.0\% manganese, and 0.5 - 1.5\%
magnesium \autocite[p.~102-103]{wardlaw33}},
except in cases where parts were required to be more resilient to daily
wear, in which case stronger magnesium alloys were used to reduce the
need for maintenance \autocite[p.~15]{guardia14}.

\subsection{Control Surfaces}

\section{Testing, Production, Variants}
All Stuka variants featured reversed gull wings, a 'broken nose' [reference]
hump to the fore of the cockpit, and fixed landing gear with aerodynamic
spatting\autocite[p.~4]{curry88}, lending the
appearance of [quote, reference]. % reasons for these three design
% choices

As a wartime plane, the Stuka had a number of evolutionary variants as
Luftwaffe requirements changed during the course of the war. Principal among these were the Ju 87A
(pre-war prototype), Ju 87B (early war, Battle of Britain), Ju 87D
(late war, improved performace), and Ju 87G (anti tank close support
role).
\subsection{Ju 87A}
The 87A went through a number of iterations in the lead up to WW2,
until it was eventually replaced by the Ju 87B production design. 

{image of Ju 87A}
\subsection{Ju 87B}
\subsection{Ju 87D}
\subsection{Ju 87G}
\section{Aeronautic Performance}
\subsection{Lift}
\subsection{Stability}
\subsection{Takeoff}
\section{Role Performance}
When calculating operational parameters for the Stuka, it is necessary
to consider that the Luftwaffe deployed the Stuka in the Western and
Eastern fronts, as well as the Desert and Meditteranean theatres.
Accordingly, calculations have been done for the most extreme
temperatures where the Stuka operated, as well as the more temperate
Battle of Britain temperatures, in order to demonstrate the versatility
of the Stuka.

Lowest temperature of operation (Moscow): -45C \autocite{raus03}
European / Battle of Britain temperatures:
Highest temperature of operation ():

% calculate atmosphere for Eastern, Western, Desert theatres, over land
% as well as sea

\printbibliography
\end{document}

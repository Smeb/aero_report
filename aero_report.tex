\documentclass[a4paper, fontsize=11pt]{scrartcl} % A4 paper and 11pt font size
% \usepackage{showframe} % page debugging
\usepackage[bottom=1.0in]{geometry} % slightly shorter margin bottom
\usepackage{times}
\usepackage[T1]{fontenc} % use 8 bit encoding that has 256 glyphs
\usepackage[english]{babel} % English language/hyphenation
\usepackage{csquotes}
\usepackage{parskip}
\usepackage{sectsty}
\usepackage{enumitem}
\usepackage{textcomp}
\usepackage{url}
\usepackage[
  backend=biber,
  style=alphabetic,
  citestyle=authoryear
]{biblatex}
\addbibresource{aero_report.bib}

\allsectionsfont{\normalfont\scshape}
\setlength{\headheight}{7.6pt}
\setlength{\headsep}{0pt}
\setlength{\parsep}{0pt}
\setlength{\topskip}{0pt}
\setlength{\topmargin}{0pt}
\setlength{\topsep}{0pt}
\setlength{\partopsep}{0pt}

%--------------------------------------------------------------------------------
%   TITLE SECTION
%--------------------------------------------------------------------------------

\newcommand{\horrule}[1]{\rule{\linewidth}{#1}}

\title{
  \normalfont \normalsize
  \textsc{ucl, computer science}
  \horrule{0.5pt} \\[0.4cm]
  \huge{Case Study: Junkers 87 'Stuka'}\\
  \horrule{0.5pt} \\[0.5cm]
}

\author{Ben Ryves}

\date{\normalsize\today}

\begin{document}
\maketitle

% abstract



% development of bombers - need for precision
% development of dive bombers
\section{Abstract}
This case study examines the Junkers Ju 87 'Stuka', a German dive bomber
designed and deployed during the Third Reich. The study begins by giving
context for the Stuka, and an examination of the motivations behind dive
bombing, before proceeding to an in depth analysis of the contstruction
and aeronautic performance of the Stuka. Where possible primary source
material such as the surviving Stuka at RAF museum Hendon, and German
maintenance manuals have been used. All other evidence has been
gathered from secondary historical commentary; and all calculations have
been informed by technical reports and textbooks.

When calculating operational parameters for the Stuka, it is necessary
to consider that the Luftwaffe deployed the Stuka in the Western and
Eastern fronts, as well as the Desert and Meditteranean theatres.
Accordingly, calculations have been performed for the hottest and
coldest temperatures at which the Stuka operated, as well as using
meteorological data from the Battle of Britain as a model for ideal
operational conditions.
\section{Junkers Ju 87}

The Junkers Ju 87 design, the development of
which was lead by Hermann Pohlman, began development in 1933. First flown
in 1935 \autocite[p.~9]{weal97}, the Stuka succeeded a previous
Junkers Ju 47 K dive bombing design, which had been rejected by the
German Reichswehr (defense ministry) as too expensive, and replaced the
Heinkel He 50 as the dive bomber of choice for the Luftwaffe. Heinkel
had also been developing a competing design, the 118, but failed to win the
contract when their design could not demonstrate the prequisite
ability to dive at a 90\textdegree\ angle, disintegrating during flight
\autocite[p.~68-69]{killen67} and forcing the pilot and judge of the
competition, Ernst Udet, to bail out.

The dive-bomber design emerged as a solution to the need for precision
bombing of tactical objectives during the later stages of
WW1, and was eventually replaced by improvements in bomb sighting,
increased bomb payloads, and guided weaponry. Dive bombers would align themselves
laterally before engaging in a dive towards a target,
descending to a given height to release their payload and then pull
out of the dive. Diving aligned the velocity of the aircraft in the
direction of the target, and reduced the distance travelled from release
to impact; this significantly reduced targeting complexity, and
correspondingly shrank the circular error probable (CEP), the
region in which 50\% of munitions were predicted to land. In comparison,
horizontal bombers using unguided bombs had to compensate for a parabolic
bomb trajectory where a bomb had horizontal velocity at launch and was
acted upon by drag and gravity during flight. Even in
calm weather, using tachometric bombsights, horizontal bombers during
WW2 had large CEPs and could not accurately hit small targets, being
more suited to mass scale interdiction bombing 'area-denial' sorties.

The increased accuracy offered by dive bombing was tactically
significant, enabling both close air support
of ground forces in combined arms operations without risk to engaged
units, as well as accurate attacks against shipping which had been
difficult to accomplish with interdiction bombing; however, diving
towards targets placed aircraft at increased risk from surface fire, and
steep diving maneuvers limited payload weight and placed increased
stress on the craft. Dive bombers were also targets of opportunity for
enemy fighters, since their diving manoeuvres were predictable and broke from
protective formations, and they could not match the manoeuvrability or
speed of fighter craft; this, in conjunction with improvements in
bombing technology, lead to the decline of the dive bomber after WW2.
\subsection{Payload}
\section{General Design of the Ju 87}

\subsection{Structure}
\subsubsection{Body}
The Stuka was a twin seater monoplane, with the fuselage acting as an
anchor for a single spar which passed through both wings and served to
distribute the load caused by lift.
This design, pioneered by Hugo Junkers in 1915 \autocite{nasa4}, provided sufficient
strength to support a single fixed wing design, which superseded
biplane designs which were hampered by aerodynamic interference
between the wings, as well as drag caused by structures
necessary to support the wings \autocite[p~.37]{peery12}.
The Stuka's fuselage was constructed from two oval
chassis sections, which were joined using rivets along the longitudinal
axis.  Internally, each section was attached to a series of longerons
which ran the longitudinal length of the Stuka, and which were attached
to a series of u-shaped frames. The use of longerons as opposed to
stringers served to reinforce the Stuka for the increased load
experienced during pullout from the diving maneuver \autocite{peery12}.

\subsubsection{Wings}
The Stuka had low mounted polyhedral wings which were formed from two spars, and consisted of
three sections; a central section and a port and starboard section. The
central section was depressed to an anhedral angle of 8 degrees, while
the two other wing sections had a dihedral angle of 12
degrees. The total wingspan was 13.8m, giving a total wing area of
31.90m\textsuperscript{2}; the length of the anhedral
part of the wing on one side of the Stuka was 1.05m, while the dihedral part was
5.5m (approximate) \footnote{No exact values could be found for wing
sections, so lengths were calculated based on known wingspan and cockpit
width, and proportions from scale drawings found in Junkers manuals
\autocite{manual39}}. The justification for using an inverted gull-wing
is not clear from documentation, since the polyhedral
design complicated manufacture and added weight; however, a similar wing
configuration can be found in the 1940 F4U Corsair, where the polyhedral
design allowed shorter landing gear while still providing clearance
for a 4m propeller \autocite{usni} - the Stuka's own propeller was 3.5m.
As the landing gear of the Stuka were fixed, reducing their length would reduce
drag. Another effect of the design was to cause the low
mounted wings to protrude from the fuselage at a perpendicular angle,
which would reduce interference between the wing and the body
\autocite[p~.203]{hartshorn31}. A final reason for the design could have
been to improve the pilot's visibility of the ground; the cockpit was
located inline with wings of the Stuka, and the low mounting and
anhedral angle would act to remove the wings from the pilot's field of
view.

In order to calculate lift, stability, and control, it will be necessary to
calculate the Equivalent Dihedral Angle, or EDA given the polyhedral
wingshape. 

\subsubsection{Control Surfaces}
Beneath the trailing edge of the wing were hinged full-span
ailerons which were operated using hydraulics. To fit the wing, these
ailerons were also divided into three sections.

\subsubsection{Materials}

The body itself was made principally from
duralumin, an aluminum alloy composed from copper, magnesium, and
manganese\footnote{Unfortunately, precise values for the alloy could not
  be located, but the general composition of duralumin is 93.5 - 95\% aluminum, 4\%
  copper, 0.5 - 1.0\% manganese, and 0.5 - 1.5\%
  magnesium \autocite[p.~102-103]{wardlaw33}},
  except in cases where parts were required to be more resilient to daily
  wear, in which case stronger magnesium alloys were used to reduce the
  need for maintenance \autocite[p.~15]{guardia14}.

  \subsection{Control Surfaces}

  \section{Testing, Production, Variants}
  All Stuka variants featured reversed gull wings, a 'broken nose' [reference]
  hump to the fore of the cockpit, and fixed landing gear with aerodynamic
  spatting\autocite[p.~4]{curry88}, lending the
  appearance of [quote, reference]. % reasons for these three design
% choices

  As a wartime plane, the Stuka had a number of evolutionary variants as
  Luftwaffe requirements changed during the course of the war. Principal among these were the Ju 87A
  (pre-war prototype), Ju 87B (early war, Battle of Britain), Ju 87D
  (late war, improved performace), and Ju 87G (anti tank close support
  role).
  \subsection{Ju 87A}
  The 87A went through a number of iterations in the lead up to WW2,
  until it was eventually replaced by the Ju 87B production design.

  {image of Ju 87A}
  \subsection{Ju 87B}
  Motor : Jumo 211 A/1
  \subsection{Ju 87D}
  \subsection{Ju 87G}
  \printbibliography
  \end{document}
% maximum diving speed 540 km/h \autocite[p~.56]{manual39}
